\section{Concurrency}
As stated in section \dots, the BW4T-Server needs to accept connections from
various BW4T-Clients.  The following decisions had to made regarding client
priorities:

\begin{itemize}
  \item Every connected client is allowed to start the debugging process and/or
    control it.
  \item The agent's actions, percepts, etc. which are time dependent are
    offered by the server in a first-come first-serve basis.
  \item The agents should be indistinguishable and should not suffer from any
    effects due to being from a different client.
\end{itemize}

Concurrency will mostly be handled by a tick system. Every time a tick is
executed, an agent performs a certain action which is allowed by the server
within that specific time-frame. If a non-shareable atomic action is requested
by an agent, the server will look up in the configuration what it is supposed
to do. The default behaviour in this situation will be that the server picks
one of the agents randomly and allows them to perform the action.

The key here is that each action is assigned a specific timing(number of
ticks). In the previous versions of BW4T, it would still be possible for a
client to complete an action even though the debugger had halted the game. By
splitting up every action in separate ticks, the world should be much more
predictable and should enhance debugging simplicity.
