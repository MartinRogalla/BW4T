\documentclass{article}
\usepackage{graphicx}

\begin{document}

\title{Retro}
\maketitle

\section*{Klant review}
Heel blij met demo. Alleen: 
Koen weet niet wat hij nu precies van ons moet verwachten. 
Alles was mooi. 

\subsection*{Groep 1}
\paragraph*{Afgelopen week}
Sonar issues gefixt, client gui drop down menu gefixt. Logger verbeterd, bijna klaar. Test geschreven voor messaging in client (ook een beetje omgeschreven). Checkstyle is aangepast, zodat er maar 1 algemene file wordt gebruikt (zijn ook wat regels aangepast zoals dat strings vaker mogen terug komen)
\begin{center}
\emph{Tip, Checkstyle weet niet alles, blijf zelf nadenken!}
\end{center}
\paragraph*{Volgende week}
nader te bepalen. 

\paragraph*{Issues} Waren er een paar, maar die zijn binnen de groep opgelost. 

\subsection*{Groep 2}
\paragraph*{Afgelopen week}
Bezig geweest met botstore en handicaps, tests geschreven, integratie met groep 3 gedaan, zodat je de bots echt kan aanmaken. Ontwerp gemaakt voor de environmentstore. Wordt opgestuurd naar de klant. Nog geen tijd gehad om de gui te maken. 

\paragraph*{Volgende week} 
Bezig met de gui van de environmentstore

\paragraph*{Issues}
Integratie/merge viel tegen

\subsection*{Groep 3}
\paragraph*{Afgelopen week} Scenaria editor afgemaakt, begonnen aan batch runs, batch runs valt mee. Veel tijd kwijt aan integratie. Botstore/epartner gefixt. 

\paragraph*{Volgende week}
Human player + bat runs

\paragraph*{issues}
Solved

\section*{Overige opmerkingen}
\subsection*{Martin - merging} 
Zorg dat je makkelijk kan merge. Misschien vaste tijd afspreken? Donderdag ochtend 9 uur wordt er gemerched. 

\subsection*{Valentine}
Gister avond veel doorgewerkt. Client was volledig vergeten. Dat was veel werk en het is handig voor de environment store en humanbotstore om de client server en core bijhouden. De structuur werd misschien niet helemaal goed begrepen door iedereen dus er komt een presentatie vanuit groep 1. Gebeurt vrijdag ochtend, na de bachelli meetings en wordt gefilmd.

\subsection*{Joop}
\begin{itemize}
	\item Veel bezig geweest met communiceren. Liep er tegen aan dat hij het moeilijk vond om duidelijk te krijgen over wat we moeten afleveren en wat de klant precies wil. De klant was niet duidelijk en wij stelde niet specifieke genoegen vragen. Joop is vervolgens specifieke vragen gaan stellen en heeft daar ook goede informatie uit gekregen.
	\item Langs koen gegaan voor batchrun. Scenario editeror heeft mas2g exporter.
	\item Katelijn gesproken over hoe het project gaat. In principe vond ze het goed, maar hoe met het straks verder? Niet alleen documentatie maar ook verklaring van keuzes, waaom? 
	\item Leerdoelen: iedereen moet dit doen. Helpt ook bij verdelen van taken. 
	\item Werkuren: Op de whatsapp was veel onenigheid over tijden. Zorg dat je vanaf nu de 28 uur per week haalt. 
\end{itemize}
  
\subsection*{Algemeen}
\begin{itemize}
	\item Scrum is niet geweldig gegaan, zijn een aantal dingen flink mis gegaan. Hoe gaan we er voor zorgen dat niet zoiets als een hele client vergeten gaat worden: \begin{center}
	{Steffan: "zorg dat de commitment er blijft en dan je elke keer een product blijft afleveren" }
	\end{center}
\item Javadoc: author lijkt niet handig volgens martin. Levert alleen maar problemen, waar leg je de grens? Xander wil het liever niet omdat we begonnen zijn met code van andere mensen.  We hebben besloten om geen autor's te doen.
\end{itemize}



\end{document}