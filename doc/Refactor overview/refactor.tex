\documentclass[11pt,a4paper]{article}
\usepackage[english]{babel}
\usepackage[utf8]{inputenc}

\title{BW4T - Refactoring}

\date{\today}

\begin{document}

\maketitle

\section{Hoe het nu is}
UML diagram \\
Spaghetticode \\
Lastig te begrijpen \\
Slechte tot geen documentatie \\
Geen testen \\
Klein beetje repast, maar veel zelf geimplementeerd \\
Het leek beetje op client server maar is toch net niet \\
Download heel groot \\
Server, client moet alles apart opstarten \\

\section{Wat hebben we tot nu toe bereikt}
Repast plugin gemaakt. minder ruimte (nog maar 8 mb) jar files eruit, depend \\
Nieuwe versie repast \\


\section{Repast}
Leuk maar niet te doen \\
Of geen repast maar alles zelf, of repast maar dan accepteren dat niet alles precies werkt zoals dat nu is zonder verandering van GOAL \\

\section{Wat kunnen wij aankomende weken nog bereiken}
Server client niet apart opstarten, maar launcher \\
Documentatie fixen \\
Testen maken \\
Lijntjes knippen \\

\section{Wat zou er kunnen gebeuren}
Alles omgooien, repast gebruiken en daarbij GOAL ook waarnodig aanpassen \\
Belangrijk is om dan goed overzicht te bewaren zodat zo als nu is nooit meer kan worden \\

\end{document}
