\documentclass{article}

\begin{document}

\title{Vergadering 09/05/2014}
\author{Contextproject: Crisis management}
\date{Afwezig: Tom \& Daniel}
\maketitle 

\section{Opening} 
11:29

\section{Agenda}
\begin{itemize}
\item Spelling nakijken
\end{itemize}

\section{Mededelingen}
\begin{itemize}
\item Daniel afwezig i.v.m. open dag
\item Tom heeft zich verslapen en gaat thuis aan het werk
\end{itemize}

\section{Hoe loopt het?}
\begin{itemize}
\item Het begint eindelijk een beetje op gang te komen. We weten nu hoe het gepland en verdeeld is.
\item Communicatie en afspraken gaan wel goed, maar er zijn te weinig dingen om met de hele groep aan het werk te zijn. Als er nu mensen klaar zijn met hun taak kunnen ze verder niets doen, omdat er niet met 6 man aan 1 ding gewerkt kan worden.
\item Google agenda werkt prima, geen klachten.
\item Steffan:\\
- Positieve punten bespreken: kijken wat er goed gaat en aangeven wat er mis gaat.\\
- Miscommunicatie: afspreken wanneer er iets ingeleverd kan worden, hoe zorgen dat het wel goed gaat.\\
- Neem 5 minuten om met het eigen team bespreken hoe de samenwerking verloopt:
\begin{itemize}
\item Groep 1: Mensen waren afwezig zonder reden. Dit moet van tevoren gemeld worden.
\item Groep 2: Heeft nog niet veel met de hele groep gewerkt, maar het ging vandaag wel goed.
\item Groep 3: Communicatie is goed, maar Trello moet beter bijgehouden worden.
\end{itemize}
- Kernwoorden: toewijding aan het team, tijd beschikbaar stellen, goed communiceren, laat weten waar je bent/wat je doet, zorg dat je beter gaat werken, pak je eigen taken op en maak het voor het team af. Kijk zelf of je aan die kernwoorden voldoet.\\
- Als er iemand minder werk dan de rest doet moet het eerst met elkaar besproken worden, anders kan je naar de contextbegeleiders stappen.
\end{itemize}

\section{Tijden Scrum planning en Scrum review/retrospective}
Bacchelli wil de scrum review op donderdag hebben en de scrum planning op vrijdag. Onze eigen planning moet dus veranderd worden.\\
Vrijdagochtend hebben we meetings met Bacchelli, daarna houden we een eigen meeting om de planning voor de volgende week te maken.\\
Donderdag komen we samen voor de review en retrospective. Bij de scrum review met de klant (11.45-12.30) hoeft niet iedereen erbij te zijn, maar wel ten minste 1 iemand uit elke groep + de product managers. De retrospective doen we in de pauze (12.30 - 13.45).\\
Elk team moet nu al een planning maken voor volgende week.

\section{Aanwezigheid}
\subsection{Op tijd komen / Niet komen}
Als je er niet kan zijn/te laat bent, moet je dit van tevoren laten weten. Afwezig zonder goede reden = taart.

\subsection{SA's}
De MAS SA's kunnen gewoon met hun eigen groep buiten de projecturen werken.

\section{Steffan}
Zie punt 4.

\section{Code review}
Kan door middel van pull requests in github. De groep die met een taak bezig is moet eerst zelf reviewen voordat er een pull request ingediend wordt.

\section{Google doc timesheet}
Er was afgesproken dat iedereen zijn werktijden ging bijhouden, maar niet iedereen heeft het goed bijgehouden.\\
Iedereen moet na deze vergadering beginnen met bijhouden. Alle uren tellen mee, dus projecturen en "extra" uren.\\
Xander gaat kijken of Toggl handig is om te gebruiken.

\section{Bowlen}
Is gereserveerd van 2 tot 4. 4 mensen kunnen niet mee. Shirley neemt knaek mee.

\section{Wvttk}
\begin{itemize}
\item Deadlines opdrachten (nakijken/inleveren)\\
De groep die met een taak bezig is moet eerst zelf nakijken, daarna de rest.

Woensdagavond 6u moet het op github staan, voor donderdagochtend moet er feedback worden gegeven.

Iedereen (behalve mensen die eraan hebben gewerkt) ten minste 1 opmerking per verslag. Joop levert alles in, behalve de individuele groepverslagen
\end{itemize}

\section{Rondvraag}
Er werd gevraagd hoe de user stories er precies uitgewerkt moeten worden. De know how stories mogen gewoon een blok tekst zijn, de rest moet volgens de template.

\section{Sluiting}
12.24


\end{document}