\documentclass{article}

\begin{document}

\title{Personal learning goals}
\author{Context project: Crisis management}

\maketitle

All the teammembers will set their personal learning goals on this page. This will be helpfull for your own reflection and can be helpfull for teammembers, so they can take it into account when dividing tasks. 
\section*{Learning goals} 

\subsection*{Wendy}
\begin{itemize}
	\item Met \LaTeX{}  leren werken
	\item met GitHub leren werken
	\item Zoveel mogelijk 'project-ervaring opdoen, deze ga ik volgend jaar voor mijn minor goed kunnen gebruiken 
\end{itemize}

\subsection*{Joop}
\begin{itemize}
	\item Verbeteren van test skills
	\item Project manager vaardigheden verwerven
	\item Goed overzicht krijgen van hoe projecten in elkaar zitten
\end{itemize}
\subsection*{Katia}
\begin{itemize}
	\item Leren een bestaand project te begrijpen en deze aan te passen/uit te breiden
	\item Samenwerken met een grote groep
	\item Werken met github
\end{itemize}

\subsection*{Arun}
\begin{itemize}
  \item Leren in grote dev. groep te werken
  \item Leren bestaande code te begrijpen en uitbreiden
  \item Werken met GitHub
  \item Product naar klant wensen maken 
\end{itemize}

\subsection*{Seu Man}
\begin{itemize}
	\item \LaTeX
	\item GitHub
	\item SCRUM
	\item Samen werken met een grote groep
	\item Verbeteren/uitbreiden van bestaande code
\end{itemize}

\subsection*{Nick}
\begin{itemize}
	\item \LaTeX
	\item GitHub
	\item JUnit
\end{itemize} 
Mijn eerste leerdoel is om \LaTeX{} te leren gebruiken. Ik heb het namelijk nog nooit gebruikt en het lijkt me toch wel handig. Vervolgens is mijn tweede leerdoel is om met GitHub te leren werken. Ook dit heb ik nog nooit gebruikt,  maar het zou wel iets makkelijker moeten zijn omdat het overeenkomsten heeft met SVN wat ik wel al
een aantal keer gebruikt heb. Ook zou ik wel wat willen leren over software testen d.m.v. JUnit tests. Ik vind dit persoonlijk niet echt interessant maar het lijkt me wel handig om een keer gedaan te hebben, zeker als het mij kan helpen om  meer bugs te voorkomen.

\subsection*{Sander}
\begin{itemize}
	\item Samenwerken in een grote groep.
	\item Gebruik van GitHub en het werken met branches.
	\item Leren met dependencies te werken binnen programma's.
	\item Hoe een Client-Server Architecture nou echt werkt.
	\item Met Scrum werken en hoe goed tijden in te schatten.
	\item Leren zeggen wat me dwarszit over andere teamleden (en hun taken).
\end{itemize}
\subsection*{Tim}
\begin{itemize}
	\item Leren hoe in groepsverband effectief gewerkt moet worden aan een groot software systeem.
	\item Communicatieve vaardigheden verbeteren.
	\item Scrum leren en inzetten als ontwikkelmethode voor software.
	\item Wat in college geleerd is over software design en software testen inzetten in de praktijk.
	\item Leren werken met Git.
\end{itemize}
\subsection*{Joost}
\begin{itemize}
	\item Learn to work in an effective way within a large group.
	\item Learn about the different roles within a Scrum team and what it's like to take the lead as Scrum Master.
	\item Learn to use GitHub.
	\item Learn more about different testing techniques and continuous integration.
	\item Learn how to understand the needs of the client in a way that works for both the client and the development team.
\end{itemize}
\subsection*{Sille}
\subsection*{Shirley}
\subsection*{Tom}
\begin{itemize}
	\item Samenwerking in een grote groep
	\item Leren GitHub te gebruiken
	\item Meer JUnit ervaring
	\item Leren met Scrum te werken
\end{itemize}
\subsection*{Valentine}
\begin{itemize}
	\item How to work together as a team.
	\item How to apply Software Engineering principles to a project.
	\item How to use Design Patterns.
\end{itemize}

\subsection*{Xander}
\begin{itemize}
	\item Leren omgaan met de veranderingen in een SCRUM team
	\item Leren omgaan met mensen die niet weten wat ze willen/zeggen
\end{itemize}

\subsection*{Ruben}
\begin{itemize}
	\item Met Github leren werken
	\item Leren om een project goed uit te voeren met behulp van Scrum
\end{itemize}
\subsection*{Calvin}
\begin{itemize}
 	\item \LaTeX
	\item Samenwerken in een groep (als "bedrijf")
	\item SCRUM
	\item Architecture design and  implementation
	\item Unit Testing
\end{itemize}
\subsection*{Martin}
\begin{itemize}
  \item How to set up a well organized testing environment with Jenkins, Maven and Sonar.
  \item How to manage a large project consisting out of many team-members in a versioning-system like Git.
  \item Working in a team with the SCRUM method.
  \item Unit testing and creating power mocks in order to test static system classes.
  \item Communicating within a team and cooperate together with a client to create a smooth transition at project delivery.
\end{itemize}
\subsection*{Jan}
\begin{itemize}
	\item How to work together in a team on the same project commiting to the same files.
	\item How to refactor a large project, specifically how to decide what are necessary changes.
	\item Applying the SCRUM Method.
	\item How to organize a team as a Scrum Master.
	\item How to help people without doing the work for them.
\end{itemize}
\subsection*{Daniel} 
\begin{itemize}
	\item how to document well. I want to achieve this by learning \LaTeX  and by doing most of the  minutes of the group process. 
	\item How to use GIT. Im working on an another project and it's really difficult to get a clear overvieuw wiitth svn. Git will be much better. 
	\item How to mantain code (testing, metrics,checkstyles). I've never coded like that and it's something that's important for every other project i'll do in the future. 
	\item How to use scrum. I have had many lectures about scrum and became really interested in it, but to really be able to do it you have to practice and experience it. 
	\item How to dive into the code. Many times i did a project there where many things i didn't undertand and i stepped away of the code and coding, because there where people who did understand. This time i really want to understand how the things work and how they work togheter so that i can learn to code.  
\end{itemize}

\end{document}
