\documentclass{article}

\begin{document}

\title{Vergadering 3}
\author{Context project: Crisis management}
\date{28-04}
\maketitle 

\section{Opening} 
11:43
\section{Goekeuring notulen}
Geen opmerking, niet iedereen had de mail gekregen. Notulen zijn goedgekeurd. 
\section{Agenda} 
Geen opmerkingen 
\section{Mededelingen} 
\begin{itemize}
\item Er zullen verneuk dingen gaan komen in de notulen, zodat iedereen ze gaat lezen. 
\item Docenten komen niet in de mail van BW4T op verzoek van Koen
\item Valentine heeft mail adressen nodig van xander + shirley
\end{itemize}
\section{Amsterdam 2 mei 9:00} 
We gaan op bezoek bij SIG om het over het project te hebben en ons de mogelijkheid te geven vragen te stellen over \emph{refactoren}. Het adres is onbekend, volgt nog een infomail over. We zullen vroeg vanuit delft vertrekken met zowel trein als auto, omdat niet iedereen week ov heeft en een auto dan goedkoper is. Er is een discussie of het handig en nodig is om met iedereen te gaan. De conclusie is dat het i.i.g. niet praktisch is qua vervoer en kosten en de mensen die niet mee gaan tijd kunnen stoppen in andere delen van het project. Groep 1 zal dit oppakken en zorgen dat het gesprek vrijdag soepel kan verlopen. Vrijdag zullen de 2 groepen (degene die naar Amsterdam gaan en degene die in Delft blijven) elkaar inlichten over de ontwikkelingen van die dag.\linebreak

Misschien handig om later te weten, mensen met een \textbf{week ov} 
Valentine, Tim, Katia, Nick, Joost, Tom, Sander, Shirly, Seu Man, Calvin.

\section{Delivarble 8 mei : Draft product vision} 
Er moet een gesprek komen met de stakeholders om wat meer specificaties te krijgen en om te overleggen of het goed is als we 1 compleet document maken i.p.v. per groep een aparte. We gaan proberen onze \emph{final} af te hebben op \textbf{8 mei}, feedback evalueren we later. 

\section{Planning} 
We gaan volgende week een test print draaien. We gaan ervan uit dat we 1 team zijn. De test sprint zal gericht zijn op de "verslagen" waaronder vallen: \begin{itemize}
\item Product planning
\item Emergent architeture
\item Product vision 
\end{itemize}

\section{Taal gebruik}
Intern zal er gebruik worden gemaakt van Nederlands. Alles wat naar externe gaat zal Engels zijn. 

\section{Gesprek Stakeholders} 
Joop en Valentine zullen dit gesprek ingaan. Er zijn verschillende zaken die meegenomen kunnen worden in dit gesprek: 
\begin{itemize}
\item wat is de huidige situatie en wat is er mis mee? : Aangeven wat wij hebben gevonden. 
\item Welke punten willen jullie verbeterd zijn en wat is het \textbf{belangrijkste?}
\item Wat zijn de belangrijkste dingen? 
\item Kunnen ze bij aankomende meetingen, betreffende de sprint review, zijn? 
\item Probeer dingen vast te stellen over het refactoren, neem geen genoegen met zoek het zelf maar uit. 
\end{itemize}

Verder werd er genoemd dat volgende meetings een timeslot moeten krijgen om zo efficiënt mogelijk te werk te gaan. 

\section{W.V.T.T.K.}
Geen w.v.t.t.k. 

\section{Rondvraag}
Geen vragen

\section{Volgende vergadering}
Vrijdag aan het eind van de ochtend een overleg van 11:30-12:30, mocht dit niet mogelijk zijn doen we het in/na de lunch. We zullen 30 min gebruiken per groep om de ontwikkelingen door te spreken en daarbij nog 15 min  om te overleggen over volgende week.  

\section{Sluiting}
12:30

\section{Toezegging}
Joop maakt aan het eind van het project een compilatie van \emph{selfies}.

\section{Actiepunten}
\begin{itemize}
\item Daniel: brum regelen bij CH-bestuur. 
\item Joop: Vragen naar parkeerplek bij SIG? 
\item Joop + valentine: Contact opnemen met stakeholders voor gesprek en gesprek verder voorbereiden. 
\item Xander: vragen uit het document voor de stakeholders naar joop mailen 
\item Joop: Kateline mailen dat we niet met iedereen naar Amsterdam gaan
\item Iedereen: goal plugin eclipse en zorgen dat je alles werkend hebt. Anders mailen 

\end{itemize}





\end{document}